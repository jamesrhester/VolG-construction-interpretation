\subsection{Developing a DDLm dictionary}

\subsubsection{Overview}

The following sections outline the typical steps that need to be taken
in order to create a high-quality dictionary suitable for incorporation
into the IUCr dictionary collection. For convenience in the following
discussion, data names are classified into three groups: ``derived'',
''observed'', and ``identifiers''. ``Identifier'' data names
serve purely to identify some object or concept in the sense of distinguishing
one thing from another, and carry no other meaning. ``Derived''
data names have values that can be calculated from the values of other
data names. ``Observed'' data names have values that are directly
measured or observed. Data names that describe aspects of an experiment,
including users, dates and equipment layout, are considered ``observed''.

The initial dictionary development process should be conducted with
the aim of describing all data names associated with the given experimental
setup or theoretical framework. At this initial stage, how data might
be arranged in data blocks should be ignored. By adopting this approach,
it is easier to later clearly define the natural association between
a data block and a single experiment or dataset.

\subsubsection{Listing candidate data names}

The concepts that might be present in a data file that the new dictionary
will describe are first listed, with suitably precise definitions.
Some useful sources of potentially relevant data names are:
\begin{enumerate}
\item Nouns present in the definitions, which will often correspond to identifiers.
For each of these identifiers, data names that describe properties
of those identifiers can also be considered
\item Key data names, which should be provided for all non-identifier data
names.
\item Data files already used in the target discipline. Every scientifically
useful value in such files can be assigned to a data name. The context
in which these values appear will contain information about the key
data names. For example, in hierarchical files one or more of the
nodes of the hierarchical tree above a value often correspond to key
data names.
\end{enumerate}

\subsubsection{Characterising data values}

Each of the data names identified in the previous step should be classified
as identifiers, observed or derived. Following this, (i) one of the
data value types listed in <DDLm type.contents entry> is assigned;(ii)
the key data names are listed; (iii) for any derived data names, the
definitions are revised, if necessary, to clarify how the values are
derived from the values of the key data names. Well-defined terminology
from the field, and references to literature can be used to keep the
definition short.

\subsubsection{\label{subsec:Increasing-computational-usefuln}Improving computational
usefulness and flexibility}

An important reason for using a restricted language to write dictionaries
is to convey information in a way that is easily manipulable by computer.
Economy of expression both in data files and dictionaries is an additional
consideration, as well as maximising flexibility for future expansion.
The following guidelines help to achieve these goals:
\begin{enumerate}
\item Any data name of type 'Text' which contains information of value for
computations should be reworked to restrict possible values to a set
of alternatives (\cifname{_enumerated_set.*}). A separate data name
can be defined to contain free-form text where information not otherwise
presented can be provided. CIF dictionaries use 'special\_details'
data names in some categories for this purpose (see, for example,
<cross ref>). 
\item Similarly, if the \cifname{_type.contents} attribute in a definition
appears as a list containing differing types, or types separated by
Boolean operators (for example \texttt{List(Real,Code)} or \texttt{Text|Real}),
the differing elements must necessarily be processed differently,
implying that there are differing meanings. Such definitions should
therefore be split into two or more data names corresponding to each
individual type.
\item Separate data names should never be used to label individual items
forming a set of identical objects, particularly where that set has
no fundamental maximum size. For example, providing data names ``gas
1'' and ``gas 2'' for gases in a gas mixture is fragile, as a three-gas
mixture might be used in the future and there is no significance to
which gas is assigned to which data name. A robust way to handle this
situation is to instead assign a label to the notional set of these
items, and then to tabulate the constituents of each combination separately.
If we label the two data names that we wish to replace as G1 and G2,
the detailed procedure is as follows:
\begin{enumerate}
\item Replace G1 and G2 in their original category with a dataname S. 
\item Create a new looped category containing a new data name SS. SS will
be used purely to identify combinations of values taken by G1 and
G2, and so can take arbitrary values. Dataname S can only take values
drawn from the values taken by SS (that is, it can only identify combinations
that have been tabulated in the SS table).
\item Create a data name GG for the new category. GG will take values that
G1 and G2 could have taken. 
\end{enumerate}
\item (This should be part of previous paragraph, not enumerated). As a
result, G1 and G2 are replaced by S and a new table appears, with
two columns labelled SS and GG. By collecting all values for GG for
which the value of SS matches S in the first table, the combination
of values referred to in the first table can be determined. Adding
a third item requires addition of a single line to the second table,
and addition of a new data name to the new table is sufficient when
adding additional information about GG in a particular mixture. Note
that both SS and GG are key data names of the new category. Figure
\ref{fig:poor-choice} is a worked example of this procedure for a
gas mixture.
\item Where there are limited choices for the value of a data name, these
choices should be specified and a number or string assigned to them.
So, for example, instead of a data name ``ion chamber location''
with the description of the position left to the author, the values
could be restricted to ``monitor'',''detector'', and ''foil''.
\item \label{enu:If-a-set}If a set of data names are expected to take values
that vary together, there is usually a benefit in moving them into
a new category and creating a key data name for them. A pointer to
this key data name will then replace them in the context in which
they originally appeared. Application of this principle effectively
reduces repetition. For example, as described above, each observed
data name has a single key data name, which might be called ``measurement
id''. However, each of the points measured in an experiment may have
identical values for items such as the list of experimenters, proposal
number, instrument and facility, and storing these values for every
measurement point is obviously inefficient. In such a case, it makes
sense to group these data names into a separate category, and create
(for example) an ``experiment id'' key data name. Strictly speaking,
this key data name would still need to be repeated for every measurement;
Subsection \ref{subsec:Data-blocks} explains how data blocks are
used to remove this requirement. \\
Similarly, when expressing the results of a modelling calculation,
many of the inputs into the model will form a set of parameters that
do not vary between calculated points. These can be grouped into a
'parameters' category and each set assigned a ``parameter id''. 
\item Software-specific data names are only useful in proportion to the
number of users with access to the relevant version of the software,
or else to the extent that a data name can be linked to a reproducible
computation through documentation or source code. The value of such
data names is likely to decline rapidly over time and they are rarely
suitable for archival use. Such definitions should be rephrased in
non software-specific terms.
\item Data names referring to the configuration of a particular instrument,
without sufficient information to allow precise reproduction of the
geometry, similarly decline rapidly in usefulness over time and away
from the originating laboratory, and are not suitable for archival
use or data transfer. Where possible, such definitions should be recast
into generally-understood terms and use made of data names for describing
geometry.
\end{enumerate}
\begin{figure}
\begin{boxedcifdefblock}
\begin{ciflisting}
loop_
   _detector.id
   _detector.gas_1
   _detector.gas_2
   _detector.gas_1_percent
   _detector.gas_2_percent
BB1    He    N     50     50
BB2    Ar    .    100      .
BB3    Ar    .    100      .
\end{ciflisting}
\end{boxedcifdefblock}

\caption{Poor choice and arrangement of data names describing detectors filled
with gas mixtures. Items \cifname{_detector.gas_1} and \cifname{_detector.gas_2}
are drawn from the same notional set, which could expand in the future.}

\label{fig:poor-choice}
\end{figure}

\begin{figure}
\begin{boxedcifdefblock}
\begin{ciflisting}
loop_
   _detector.id
   _detector.gas_mix
BB1      2
BB2      1
BB3      1

loop_
   _gas_mix.id
   _gas_mix.gas
   _gas_mix.percent
1        Ar       100
2        He        50
2        N         50
\end{ciflisting}
\end{boxedcifdefblock}

\caption{Improved version of loop in Figure \ref{fig:poor-choice} }

\end{figure}


\subsubsection{\label{subsec:Data-blocks}Data blocks}

Encapsulating data within data blocks enables single-valued data names
to be lifted outside tabulations wherever necessary in the data block
(see Section <advanced>). Therefore, in the example given in point
\ref{enu:If-a-set} above, rather than needing to include an ``experiment
id'' for every measurement point, the experiment id is assigned a
single value for the whole data block, and so the appropriate value
for each measurement point is unambiguous. Explicit mention of that
value for every point is therefore unnecessary and the data name may
be completely dropped from the data block. As a consequence, the data
block contents become restricted to data for which 'experiment id'
is constant, that is, the block contains data from a single experiment.
When constructing a dictionary, data names that will take a single
value in the data block should be placed into 'Set' categories. Any
child data names of items in these 'Set' categories need not be included
in the dictionary. Which data names to restrict in this way is essentially
dictated by the intended meaning of a data block, for example, whether
it contains a single experiment or a single scan.
